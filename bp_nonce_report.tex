\documentclass[11pt]{article}
\usepackage[margin=1in]{geometry}
\usepackage{graphicx}
\usepackage{caption}
\usepackage{hyperref}
\usepackage{amsmath}
\usepackage{booktabs}

\title{Brazilian Portuguese Nonce-Word Acceptability}
\author{David Nguyen \and Rosser Martin}
\date{\today}

\begin{document}
\maketitle

\begin{abstract}
We examine whether word structure (syllable count and initial segment type) influences nonce-word acceptability judgments by native Brazilian Portuguese speakers. We use mosaic plots and chi-square tests to assess associations.
\end{abstract}

\section{Introduction}
We test:
\begin{itemize}
  \item \textbf{H$_0$:} Word structure does not influence acceptability.
  \item \textbf{H$_1$:} Word structure does influence acceptability.
\end{itemize}

\section{Methods}
Data are loaded from \texttt{bp-nonce.csv}. Variables:
\begin{itemize}
  \item \texttt{response}: accept vs.\ reject
  \item \texttt{length}: syllable count
  \item \texttt{initial}: initial segment category
\end{itemize}

\section{Results}

\subsection{Contingency Table}
\begin{verbatim}
<<R code prints xtabs summary here>>
\end{verbatim}

\subsection{Mosaic Plot}
\begin{figure}[h!]
  \centering
  \includegraphics[width=0.8\textwidth]{mosaic_plot.png}
  \caption{Responses by length and initial segment}
\end{figure}

\subsection{Chi-Square Tests}
\textbf{Response vs.\ Initial:} p = 0.59, fail to reject H$_0$.\\
\textbf{Length vs.\ Initial:} p = 1.00, variables independent.

\section{Discussion}
No evidence that initial segment influences acceptability. Syllable count and initial are independent by design. Future work could include logistic regression of response on both factors.

\end{document}
